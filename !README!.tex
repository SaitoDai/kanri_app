ご覧くださいましてありがとうございます。齋藤大です。


■サイトコンセプト
・ターゲット:商品の売上げ管理に悩む中小企業
・目的:売上げ管理の向上
・「user(ユーザー)」は実際に使用する企業の社員、「buyer(注文主)」はその企業の顧客、「destination(納品先)」は商品の出荷先をそれぞれ想定

■こだわった点
・シンプルなUI
・Laravel Breeze無しで認証機能を実装
・商品の種類をプルダウンで切り替えると、非同期通信で在庫数や商品名が変化する
・カード内一覧のレコードが増えてきたときに便利な「注文主ごとに商品の絞り込みができる機能」を実装
・その他、各レコードが増えてきた場合を想定した絞り込み機能を実装
・「destination(納品先)」を「buyer(注文主)」に紐づけることで、レコードの管理がしやすい

■次回気を付けたい点
・migrationを途中からDB上で直接行ってしまった。次回はLaravelにファイルを作る形にして残るようにしたい。
・ローカルではできた画像のアップロードがデプロイ後はできない。自分では原因不明。


■プログラミング学習歴	
<独学(2023年1月~7月)>	
Progateを使って独学でプログラミングの勉強を始める。
HTML, CSSでビューの全体的なレイアウトを, php, JavaScriptで変数、配列、関数、クラスなどの基礎を学習する。
		
<スクール(2023年7月~2024年3月)>		
"侍エンジニア塾を利用。
独学で学習したHTML, CSS, php, JavaScriptに加えてLaravel, mySQL, gitを始める。
・独学での学習を、より深く学習(Bootstrap, モーダル, Api非同期通信など)
・Laravel Breezeを使わない認証機能の実装。
・作成したコードをgithubに上げてインストラクターと共有。
・オリジナルWebアプリはherokuでデプロイ。
・要件定義書、機能一覧表の作成。
・X(旧Twitter)風アプリ、ToDoアプリ、商品管理アプリ(完全オリジナル)を作成。
※ポートフォリオサイトURL:https://kanri-app-e0b6a6d6e512.herokuapp.com/"		

<独学(2024年4月~)>		
現在、書籍「基礎から学ぶLarave(C&R出版)」を参考にしながら、Linuxでの開発を試みる

以上